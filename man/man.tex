\documentclass[12pt]{article}

\usepackage{amssymb}
\usepackage{color}
\usepackage{amsmath}
\usepackage{hyperref}
\usepackage{listings}
\usepackage{mathtools}
\setlength{\parskip}{1em}
\setlength{\parindent}{0em}
\usepackage[margin=1.0in]{geometry}

\lstset{frame=tb,
  aboveskip=3mm,
  belowskip=3mm,
  showstringspaces=false,
  columns=flexible,
  basicstyle={\small\ttfamily},
  numbers=none,
  numberstyle=\tiny\color{gray},
  keywordstyle=\color{blue},
  commentstyle=\color{dkgreen},
  stringstyle=\color{mauve},
  breaklines=true,
  breakatwhitespace=true,
  tabsize=3
}

\begin{document}

\renewcommand{\l}{\left(}
\renewcommand{\r}{\right)}

\title{\vspace{60mm}\textbf{mms}\\Micromouse Simulator}
\author{Mack Ward}
\date{2015}
\maketitle

\newpage
\renewcommand*\contentsname{Table of Contents}
\tableofcontents

\newpage
\section{Introduction}

\subsection{Summary}

This is an application for developing and simulating Micromouse algorithms. For
information on the Micromouse competition, see the
\href{http://en.wikipedia.org/wiki/Micromouse.}{Micromouse Wikipedia page}.

\subsection{Motivation}

Back when I was in IEEE, as I'm sure is the case now, we procrastinated on
mostly all of our projects. This included Micromouse; if it was more than a
week before the competition, you could safely assume our robot wasn't built
yet. This proved to be problematic for the programmers on the project. They
weren't able to start writing code until way too late (read "the night before
the competition").

In an attempt to solve the problem of not being able to write and test code
before the robot was built, we built this simulator.

The simulator is also meant to be a teaching tool - to help those who may be
interested, but not familiar, with programming to take their first few baby
steps.

\subsection{Contributing Projects}

This section isn't finished yet. For now, see the README.md in src/lib.

% TODO: upforgrabs
% Write about all of the contributing projects here

\section{Installation}

Regardless of what platform you're planning on simulating on, there are a few
dependencies that you must install before you can start. They are as follows:

\subsubsection{\href{http://freeglut.sourceforge.net/}{freeglut}}

Allows us to create a window and draw things within it.

Note: We're thinking about replacing this with GLFW, a more modern and well kept
project that performs a very similar job.

% TODO: upforgrabs
% Include a longer description of freeglut. Possibly link to the page.

\subsubsection{\href{http://glew.sourceforge.net/}{GLEW}}

Helps with drawing things, and is required by many of the libraries that
we used to implement the simulator.

% TODO: upforgrabs
% Include a longer description of GLEW. Possibly link to the page.

\subsection{Linux (Ubuntu)}

Questions about Linux installation can be directed at \url{mward4@buffalo.edu}.

Simply open a terminal and run the following:

\begin{lstlisting}
sudo apt-get install g++
sudo apt-get install freeglut3-dev
sudo apt-get install libglew-dev
\end{lstlisting}

\subsection{Windows}

Questions about Windows installation can be directed at \url{kt49@buffalo.edu}
and \url{tomaszpi@buffalo.edu}.

Beware! You're entering into wild territory! A few people have gotten the
simulator to work on Windows, but only with some finicking. If you intend to
run the simulator on Windows, just be aware that it may take some extra effort
to get things working properly.

\subsubsection{Download freeglut and GLEW}

See http://www.cs.uregina.ca/Links/class-info/390AN/WWW/Lab1/GLUT/windows.html
Unzip them and place them wherever you desire.

\subsubsection{Setting up Visual Studio Project}

NOTE: These instructions were written for VS 2015

Once freeglut and GLEW are downloaded as per the above instruction you will
want to create a project in Visual Studio.  To do this automatically:

\begin{enumerate}

\item Open VS and select: File $\rightarrow$ New $\rightarrow$ Project From
Existing Code

\item Select Visual C++ from the dropdown and hit Next

\item Enter the following options on the subsequent screen

    \begin{enumerate}

    \item Project file location - The path to \textbf{inside} the /src folder\\
    (ex. D:$\backslash$Users$\backslash$Tomasz$\backslash$Documents
    $\backslash$GitHub$\backslash$mms$\backslash$src)

    \item Project name - The name you wish the project to have (ex. MMS)

    \item Add files to the project from these folder - Leave Default

    \item File types to add to the project - Leave Default

    \item Show all files in Solution Explorer - Checked
   
    \item Hit Next

    \end{enumerate}

\item Make sure 'Use Visual Studio' is checked and under 'Project Type' it says
'Console Application Project'.  Nothing else should be checked

\item Hit Finish and Visual Studio will assemble the project for you

\item (\textbf{Encouraged}) Check 'Show all Files' under the 'Project' menu
this will make the 'Solution Explorer' pane mirror the directory structure
   
\item In the 'Solution Explorer' right click the project file (should be the
second from the top) and select 'Properties'.

\item Under C/C++ - General - Additional Include Directories: Add the include
folders within the freeglut and GLEW folders you downloaded earlier, along with
the GL folder within the include folder for GLEW, and the lib folder within
/src.  For me it looks like this:
    
C:$\backslash$Users$\backslash$Kyle$\backslash$Desktop$\backslash$openGL$\backslash$glew-1.11.0$\backslash$include$\backslash$GL
    
C:$\backslash$Users$\backslash$Kyle$\backslash$Documents$\backslash$GitHub$\backslash$mms$\backslash$src$\backslash$lib
    
C:$\backslash$Users$\backslash$Kyle$\backslash$Desktop$\backslash$openGL$\backslash$glew-1.11.0$\backslash$include
    
C:$\backslash$Users$\backslash$Kyle$\backslash$Desktop$\backslash$openGL$\backslash$freeglut$\backslash$include
    
Click "Apply".
   
\item Under C/C++ $\rightarrow$ Output Files $\rightarrow$ Object File Name:

Enter '\$(IntDir)/\%(RelativeDir)/'.

Click "Apply".

\item Under Linker $\rightarrow$ General $\rightarrow$ Additional Library Directories:

Add the lib folders within the freeglut and GLEW folders.  For me it looks like this:
    
C:$\backslash$Users$\backslash$Kyle$\backslash$Desktop$\backslash$openGL$\backslash$glew-1.11.0$\backslash$lib

C:$\backslash$Users$\backslash$Kyle$\backslash$Desktop$\backslash$openGL$\backslash$freeglut$\backslash$lib
    
Click "Apply".
    
\item Under Linker $\rightarrow$ Input $\rightarrow$ Additional Dependencies: Add the following:

freeglut.lib
    
glew32.lib
    
Click "Apply".
    
\item Go to the folder where the glew files are held, glew-1.1x.0 (there are
different versions: 1.11 and 1.13 are proven to work).  Go the lib directory
and the following files may or may not be there:
    
glew32.lib
    
glew32s.lib
    
If they are not already in lib, go to Release$\backslash$Win32 and copy them to the lib directory.
    
\item Copy the freeglut.dll and glew32.dll files to the src folder, they can be found here:

    ...$\backslash$freeglut$\backslash$bin

    ...$\backslash$glew-1.1x.0$\backslash$bin$\backslash$Release$\backslash$Win32
    
\item In the 'Solution Explorer' right click the project and select Add
$\rightarrow$ Existing Item.  Select
mms$\backslash$res$\backslash$parameters.xml.  This allows for easy access to
the parameters file.

\item To build the project select Build $\rightarrow$ Build Solution or just run it and
you will be prompted if you want to build the changed project

\item To run the program select Debug $\rightarrow$ Start Without Debugging

\end{enumerate}

% TODO: upforgrabs
% Clean up these instructions, add bold, italics, and code formatting where
% appropriate

\subsection{Mac}

Questions about Mac installation can be directed at \url{dpclark4@buffalo.edu}
and \url{srsiegar@buffalo.edu}.

You should also bug those guys about putting actual installation instructions
here. 'Cause right now, we don't have any :(.

% TODO: upforgrabs
% Put installation instructions here

\section{Building}

\subsection{Linux}
TODO

\subsection{Windows}
TODO

\subsection{Mac}
TODO

\section{Running}

The simulation can be run by executing the binary in the "bin" directory, as in:
    
\begin{lstlisting}
    ./<path-to-bin>/MMSim
\end{lstlisting}

For example, if you are in the "src" directory, enter the following to run the
simulation:

\begin{lstlisting}
    ../bin/MMSim
\end{lstlisting}

\subsection{Keyboard Commands}

During the simulation, a number of keyboard commands are available to the use.
They are as follows (though I can't promise that this is 100\% updated):

\begin{center}
\begin{tabular}{ c | c | c }
  Key & Effect & Discrete Mode Only\\
\hline
  p & toggle (p)ause/resume & X\\
  f & make the mouse go (f)aster & X\\
  s & make the mouse go (s)lower & X\\
  l & cycle through the available (l)ayouts & \\
  r & toggle zoomed map (r)otation & \\
  i & zoom (i)n on the zoomed map & \\
  o & zoom (o)ut on the zoomed map & \\
  t & toggle wall (t)ruth visibility & \\
  t & toggle tile (c)olors & \\
  t & toggle wall fo(g) & \\
  x & toggle tile te(x)t & \\
  d & toggle tile (d)istance values & \\
  w & toggle (w)ireframe mode & \\
  q & (q)uit & \\
\end{tabular}
\end{center}

\subsection{Runtime Parameters}

TODO

\section{Writing Your Own Algorithms}


\subsection{Discrete and Continuous Modes}

TODO

\subsection{IMouseInterface}

TODO

%For users writing their own algorithms, they should call any one of the
%following functions in their code to receive input and generate output (Note
%that the commands below assume that the user has a pointer to a MouseInterface
%object, and the the variable name for the pointer is "m_mouse"):
%
%    1) m_mouse->wallFront()
%    2) m_mouse->wallRight()
%    3) m_mouse->wallLeft()
%    4) m_mouse->moveForward()
%    5) m_mouse->turnRight()
%    6) m_mouse->turnLeft()
%
%Functions 1-3 return true or false values corresponding to whether or not there
%is a wall to the front, right, or left of the robot. Functions 4-6 instruct the
%robot to move forward, turn left, or turn right. No other functions are
%provided to the user and all other functionality must be supplied else-where, 
%as is the case with the actual MicroMouse robot.

\subsection{Getting Started}

All user-defined mouse algorithms should be placed within the mouse directory,
preferably within their own directories. While it is possible to define a
single function for the algorithm, a good strategy is to define a class that
contains the appropriate functions for executing the algorithm. This will make
the algorithm easier to understand, easier to debug, and much more portable.
AlgorithmTemplate ".h" and ".cpp" files have been provided to help those who
may not know the syntax for defining classes in C++. You can simply copy these,
change the approprite identifiers, and provide definitions for the given solve
function, as well as you own helper functions.

Additionally, the algorithm class (that you define) should implement the
IMouseAlgorithm interface. This ensures that the algorithm is compatible with
the simulation utilities already defined.

Lastly, in order to actually execute a user-defined algorithm, the
"src/mouse/MouseAlgorithms.cpp" must be modified to include your code.

\subsection{Input Buttons}

TODO

\section{Implementation}

% TODO: upforgrabs
% Write about the implementation details - as much as your hear desires!

\subsection{Modern OpenGL}
TODO

\subsection{Collision Detection}
TODO

\subsection{Logging}
TODO

\subsection{Parameters and State}
TODO

%Software Components
%
%========================
%
%-------------------------------------------------------------------------------
%
%Building the Project on Ubuntu
%==============================
%
%After any changes have been made to your code, you must remake the project from
%the root directory (the project folder that contains "src", "bin", etc.) as
%follows:
%
%    make
%
%If your code doesn't seem to be working properly, try entering the following,
%again from the root directory:
%
%    make clean
%    make
%
%The use of "make clean" simply requires the computer to recompile your files, 
%despite it potentially thinking that changes have not been made.
%
%-------------------------------------------------------------------------------
%
%Building the Project on Windows
%===============================
%    
%- To build the project select Build -> Build Solution or just run it and you will
%  be prompted if you want to build the changed project
%
%- To run the program select Debug -> Run Without Debugging. Hit Yes if you want to
%  rebuild
%
%If you need to clean for some reason you can go to Build -> Clean Solution
%
%-------------------------------------------------------------------------------


\end{document}
